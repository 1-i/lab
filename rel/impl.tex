\begin{section}{Implementazione}

	Il lexer ed il parser sono stati generati automaticamente a partire
	dalla grammatica con BNFC.

	Il tool di pretty printing compie una visita dell'albero della
	sintassi astratta, memorizzando un campo addizionale di chiave
	intera per rappresentare il numero di livelli di indentazione.

	La realizzazione del type checker pone qualche questione di
	natura implementativa la cui soluzione non \`e banale. In primo luogo,
	dal momento che come da specifica il linguaggio supporta dichiarazioni
	di funzioni e variabili ad ogni livello di annidamento, \`e necessario
	affrontare il problema della visibilit\`a delle dichiarazioni
	al variare del blocco attivo. Le regole di tipo che abbiamo definito
	suggeriscono di rappresentare i dati come una pila di contesti,
	devono quindi essere definite opportune funzioni per gestire i
	cambi di contesto. 
	Le associazioni identificatore/tipo all'interno
	del singolo contesto sono gestite da una tabella di hash. In
	secondo luogo, non \`e immediato uniformare l'eterogeneit\`a tra gli
	identificatori di variabili e di funzioni, se per i primi \`e
	sufficiente memorizzare un tipo, per i secondi \`e necessaria l'intera
	segnatura. La soluzione che abbiamo implementato consiste nella
	definizione di un tipo di dato ad-hoc, che possa rappresentare sia
	un singolo tipo, sia la segnatura di una funzione. I controlli di
	consistenza sono delegati a funzioni ausiliarie.
	Il cuore del type checker \`e costituito da due funzioni, che nel
	codice hanno nome \texttt{inferExp} e \texttt{checkStm}. Come i
	nomi stessi lasciano intuire, la prima determina il tipo di
	un'espressione, mentre la seconda determina se lo statement che
	riceve per argomento sia valido o meno. Il resto del codice \`e
	composto da funzioni che svolgono bookkeeping di varia natura, come
	tenere aggiornato lo stato o gestire gli errori.

\end{section}
