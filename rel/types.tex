\begin{section}{Tipi}
	Per prima cosa definiamo l'insieme di tipi usato da \textit{ME}.
	Sia:
	\[
		T_0 = \{ \texttt{int}, \texttt{double}, \texttt{string},
			\texttt{char}, \texttt{bool} \}
	\]
	Inoltre sia $T_1$ l'insieme minimale per inclusione tale che
	$T_0 \subseteq T_1$ e che $\forall \tau \in T_1 . 
	((*\tau \in T_1) \wedge (\array[\tau] \in T_1)) $. Definiamo quindi
	$T_2 = \{ \&\tau \mid \tau \in T_1 \}$. Sia infine 
	$T = T_1 \cup T_2$. L'insieme dei tipi di \textit{ME} \`e $T$.

	Definiamo inoltre l'insieme delle segnature di funzione 
		$F = \{(\tau, P) \mid t \in T, P \in T^*\}$.

	Sia $\Gamma_0$ un contesto. Definiamo
	un \textit{ambiente} $\Delta$ come una pila di contesti; formalmente,

	\begin{itemize}
		\item $\varnothing$ \`e un ambiente.
		\item Se $\Delta$ \`e un ambiente e $\Gamma_0$ \`e un contesto,
			allora $\Delta.\Gamma_0$ \`e un ambiente.
	\end{itemize}	

	Sia $\Delta = \Gamma_n.\Gamma_{n-1} \dots \Gamma_1.\Gamma_0$ un
	ambiente, $x$ una variabile, $k$ il minimo intero per
	cui $x$ \`e in $\Gamma_k$,  allora:

	\begin{prooftree}	
		\AxiomC{}
		\RightLabel{se $x:\tau$ in $\Gamma_k$}
		\UnaryInfC{$\Delta \vdash x : \tau$}
	\end{prooftree}

	Introduciamo ora le regole per i tipi delle espressioni. Nelle
	seguenti, $\Gamma$ rappresenta un contesto.

	\begin{reg}{Tipi Elementari}
		\begin{prooftree}
			\AxiomC{}
			\RightLabel{se $x$ \`e un intero letterale}
			\UnaryInfC{$\Gamma \vdash x : \texttt{int}$}
		\end{prooftree}
	\end{reg}

	Regole analoghe per gli altri tipi elementari.

	\begin{reg}{Operatori booleani}
		\begin{prooftree}
			\AxiomC{$\Gamma \vdash x : \texttt{bool}$}
			\AxiomC{$\Gamma \vdash y : \texttt{bool}$}
			\BinaryInfC{$\Gamma \vdash x \texttt{ or } y : \texttt{bool}$}
		\end{prooftree}
	\end{reg}

	Regole analoghe per \texttt{or}.

	\begin{reg}{Operatori di confronto}
		\begin{prooftree}
			\AxiomC{$\Gamma \vdash x : \tau$}
			\AxiomC{$\Gamma \vdash y : \tau$}
			\BinaryInfC{$\Gamma \vdash x \texttt{ == } y : \texttt{bool}$}
		\end{prooftree}			
		\begin{prooftree}
			\AxiomC{$\Gamma \vdash x : \tau$}
			\AxiomC{$\Gamma \vdash y : \sigma$}
			\RightLabel{se $\exists \tau'\exists \sigma'.
					(\tau=*\tau' \vee \tau=\&\tau') \wedge 
					(\sigma=*\sigma' \vee \sigma=\&\sigma')$}
			\BinaryInfC{$\Gamma \vdash x \texttt{ == } y : \texttt{bool}$}
		\end{prooftree}			
	\end{reg}

	Regole analoghe per \texttt{>} e \texttt{<}.

	\begin{reg}{Somma}
		\begin{prooftree}
			\AxiomC{$\Gamma \vdash x : \tau$}
			\AxiomC{$\Gamma \vdash y : \sigma$}
			\RightLabel{se $\tau, \sigma 
			\in \{ \texttt{int}, \texttt{double},
				\texttt{string} \}$ }
			\BinaryInfC{$\Gamma \vdash x \texttt{+} y : \max 
				(\tau, \sigma)$}
		\end{prooftree}
		Dove $\texttt{int} \preceq \texttt{double} \preceq 
			\texttt{string}$
	\end{reg}

	Regola analoga per \texttt{-}.
	
	\begin{reg}{Moltiplicazione}
		\begin{prooftree}
			\AxiomC{$\Gamma \vdash x : \tau$}
			\AxiomC{$\Gamma \vdash y : \sigma$}
			\RightLabel{se $\tau, \sigma 
			\in \{ \texttt{int}, \texttt{double}\}$ }
			\BinaryInfC{$\Gamma \vdash x \texttt{*} y : 
			\max (\tau, \sigma)$}
		\end{prooftree}
		Dove $\texttt{int} \preceq \texttt{double} $
	\end{reg}

	Regola analoga per \texttt{/}.

	\begin{reg}{Negazione logica}
		\begin{prooftree}
			\AxiomC{$\Gamma \vdash x : \texttt{bool} $}
			\UnaryInfC{$\Gamma \vdash \texttt{!}x : \texttt{bool} $}
		\end{prooftree}
	\end{reg}

	\begin{reg}{Dereferenziazione}
		\begin{prooftree}
			\AxiomC{$\Gamma \vdash x : *\tau $}
			\UnaryInfC{$\Gamma \vdash *x : \tau $}
		\end{prooftree}
	\end{reg}

	\begin{reg}{Referenziazione}
		\begin{prooftree}
			\AxiomC{$\Gamma \vdash x : \tau $}
			\RightLabel{se $\tau \in T_1$}
			\UnaryInfC{$\Gamma \vdash \&x : \&\tau $}
		\end{prooftree}
	\end{reg}

	\begin{reg}{Array}
		\begin{prooftree}
			\AxiomC{$\Gamma \vdash x : \array[\tau]$}
			\AxiomC{$\Gamma \vdash y : \texttt{int}$}
			\BinaryInfC{$\Gamma \vdash x[y] : \tau$}
		\end{prooftree}
	\end{reg}

	Introduciamo ora regole per dedurre la validit\`a di uno statement
	rispetto ai tipi. Se $s$ \`e uno statement, abbrevieremo tale
	nozione dicendo ``$s$ valido''. Nelle seguenti, $\Delta$ indica
	un environment, $\Gamma$ indica un contesto, $S$ indica una lista di
	statement.

	\begin{reg}{Espressioni}
		\begin{prooftree}
			\AxiomC{$\exists \tau \in T (\Delta \vdash x:\tau)$}
			\UnaryInfC{$\Delta \vdash x \text{ valido }$}
		\end{prooftree}
	\end{reg}

	\begin{reg}{Assegnazioni}
		\begin{prooftree}
			\AxiomC{$\Delta \vdash e : \tau$}
			\AxiomC{$\Delta \vdash S \text{ validi}$}
			\RightLabel{$x:\tau$ in $\Delta$}
			\BinaryInfC{$\Delta \vdash (x \texttt{:=} e \texttt{;} S)
				\text{ valido}$}
		\end{prooftree}

		\begin{prooftree}
			\AxiomC{$\Delta \vdash e : *\tau$}
			\AxiomC{$\Delta \vdash S \text{ validi}$}
			\RightLabel{$x:\&\tau$ in $\Delta$}
			\BinaryInfC{$\Delta \vdash (x \texttt{:=} e \texttt{;} S)
				\text{ valido}$}
		\end{prooftree}
	\end{reg}

	\begin{reg}{Dichiarazione di variabili}
		\begin{prooftree}
			\AxiomC{$\Delta.(\Gamma,x:\tau) \vdash S \text{ validi}$}
			\RightLabel{$x$ non in $\Gamma$}
			\UnaryInfC{$\Delta.\Gamma \vdash (x \texttt{:} \tau \texttt{;}
				S) \text{ valido}$}
		\end{prooftree}
	\end{reg}

	\begin{reg}{Blocchi}
		\begin{prooftree}
			\AxiomC{$\Delta.\varnothing \vdash S_1 \text{ validi}$}
			\AxiomC{$\Delta \vdash S_2 \text{ validi}$}
			\BinaryInfC{$\Delta \vdash 
				(\texttt{do } S_1 \texttt{ end } S_2)$ 
			\text{ valido}}
		\end{prooftree}
	\end{reg}

	\begin{reg}{Iterazione e condizionali}
		\begin{prooftree}
			\AxiomC{$\Delta \vdash x : \texttt{bool}$}
			\AxiomC{$\Delta \vdash S \text{ validi}$}
			\BinaryInfC{$\Delta \vdash
				(\texttt{if } x \texttt{ then } S \texttt{ end} )
				\text{ valido} $}
		\end{prooftree}
	
		\begin{prooftree}
			\AxiomC{$\Delta \vdash x : \texttt{bool}$}
			\AxiomC{$\Delta \vdash S_1 \text{ validi}$}
			\AxiomC{$\Delta \vdash S_2 \text{ validi}$}
			\TrinaryInfC{$\Delta \vdash
				(\texttt{if } x \texttt{ then } S_1 \texttt{ else } 
				S_2 \texttt{ end}) \text{ valido} $}
		\end{prooftree}
		
		\begin{prooftree}
			\AxiomC{$\Delta \vdash x : \texttt{bool}$}
			\AxiomC{$\Delta \vdash S \text{ validi}$}
			\BinaryInfC{$\Delta \vdash
				(\texttt{until } x \texttt{ loop } S \texttt{ end} 
				) \text{ valido} $}
		\end{prooftree}

		\begin{prooftree}
			\AxiomC{$\Delta \vdash x : \texttt{bool}$}
			\AxiomC{$\Delta \vdash s \text{ valido}$}
			\AxiomC{$\Delta \vdash S \text{ validi}$}
			\TrinaryInfC{$\Delta \vdash
				(\texttt{from } s \texttt{ until } x \texttt{ loop } 
				S \texttt{ end} ) \text{ valido} $}
		\end{prooftree}

	\end{reg}

	\begin{reg}{Dichiarazione di funzioni}
		\begin{prooftree}
			\AxiomC{$\Delta.
				(\Gamma, f : (\tau_1 \dots \tau_n) \rightarrow \tau_r)).
				(\varnothing, x_1 : \tau_1 \dots
				x_n : \tau_n, f : (\tau_1 \dots \tau_n) 
				\rightarrow \tau_r) 
				\vdash S \text{ v.}$}
			\RightLabel{$\phi$}
			\UnaryInfC{$\Delta.\Gamma \vdash
				( f(x_1 : \tau_1 \dots x_n : \tau_n) : \tau_r 
				\texttt{ do } S \texttt{ end} ) \text{ valido} $}
		\end{prooftree}
		Con $\phi =$ ``$f$ non in $\Gamma$, $\{x_i\}$ a coppie distinti''.
	\end{reg}

	\begin{reg}{Applicazione di funzioni}
		\begin{prooftree}
			\AxiomC{$\Gamma \vdash f : (\tau_1 \dots \tau_n) \rightarrow
			\tau_r$}
			\AxiomC{$\Gamma \vdash x_1 : \tau_1$}
			\AxiomC{$\dots$}
			\AxiomC{$\Gamma \vdash x_n : \tau_n$}
			\QuaternaryInfC{$\Gamma \vdash f(x_1 \dots x_n) : \tau_r$}
		\end{prooftree}
	\end{reg}

	\begin{reg}{Funzioni builtin}
		\begin{prooftree}
			\AxiomC{$\Delta \vdash x : \texttt{int}$}
			\AxiomC{$\Delta S \text{ validi}$}
			\BinaryInfC{$\Delta \vdash (\texttt{writeInt(} x \texttt{);} S)
			\text{ valido}$}
		\end{prooftree}
	\end{reg}

	Regole analoghe per \texttt{writeDouble} e \texttt{writeString}.

\end{section}
